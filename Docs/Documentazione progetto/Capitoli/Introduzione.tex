L'approccio seguito per la stesura del modello dinamico del veicolo autobilanciato ha da subito preso una via meno \textit{tradizionale} rispetto al classico metodo risolutivo: abbiamo infatti preferito, data il nostro \textit{background} informatico, approcciare il problema direttamente in ambiente Matlab, sfruttando sin da subito le potenzialità di calcolo offerte dal software di \textit{Mathworks}.

Nello specifico, per la parte di stesura e definizione della dinamica, abbiamo inzialmente seguito una via risolutiva duale, portando avanti sia un'analisi letterale, sfruttando le potenzialità del \textbf{calcolo simbolico} messe a disposizione delle funzionalità di \textbf{live scripting}, sia uno studio numerico (considerando quindi le varie grandezze fisiche con i valori definiti delle specifiche di progetto).

In linea di massima lo sviluppo del progetto ha seguito un andamento a step graduali, cadenziati da incontri settimanali in cui poter confrontare e consolidare lo \textit{stato di avanzamento dei lavori}: nello specifico, il lavoro ha seguito uno sviluppo in questa direzione, step by step, rappresentabile in linea di massima da queste \textit{pietre miliari}:
\begin{itemize}
	\item \textbf{Dinamica di ogni singolo corpo rigido}: abbiamo impostato il problema della dinamica andando a considerare il veicolo auto bilanciato come un insieme di corpi rigidi di cui poterne studiare la dinamica in maniera separata;
	\item \textbf{Dinamica completa del VAB}: siamo andati poi a considerare il sistema nella sua completezza, andando ad unire i contributi dei corpi rigidi considerati in prima battuta singolarmente;
	\item \textbf{Linearizzazione}: TODO
	\item \textbf{Definizione del controllo}: prima lineare poi non lineare TODO
	\item \textbf{Discretizzazione}: TOOD
	\item TODO altri step
\end{itemize}

TODO: note varie